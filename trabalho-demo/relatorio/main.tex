\documentclass[12pt]{article}

\usepackage{sbc-template}
\usepackage{graphicx,url}
\usepackage[brazil]{babel}   
\usepackage[utf8]{inputenc}  
\usepackage[T1]{fontenc}

\usepackage{microtype}
\usepackage[caption=false]{subfig}
\usepackage{bm}
\usepackage{amsmath,amsfonts,amssymb}
\usepackage{booktabs,longtable}

\sloppy

\title{Metaheurística GRASP com refinamento por busca local para o Flowshop Permutacional}

\author{Alberto F. K. Neto\inst{1}}

\address{Instituto de Informática -- Universidade Federal do Rio Grande do Sul
  (UFRGS)\\
  Caixa Postal 15.064 -- 91.501-970 -- Porto Alegre -- RS -- Brazil
  \email{afkneto@inf.ufrgs.br} 
}

\begin{document} 

\maketitle

\section{Modelagem matemática com programação linear inteira}

O modelo utilizado foi descrito por \cite{tseng2004-flowshop-models}.

Considera $M$ máquinas, $N$ tarefas. $T_{rj} \geqslant 0$ representa o tempo de processamento
da tarefa $j$ na máquina $r$, para toda tarefa e máquina.

Variável $C_{ri} \geqslant 0$ indica o tempo que a tarefa $i$ completou na máquina $r$.

Variável $D_{ik} \in \{0,1\}$ (=1) indica se a tarefa $i$ é executada em algum momento antes da
tarefa $k$; (=0) caso contrário.

Parâmetro ``big-M'' $P$.

\begin{align}
   \text{Minimize } C_\mathrm{max}
\end{align}
Sujeito a:
\begin{align}
   & C_{1i} \geqslant T_{1i} & & 1 \leqslant i \leqslant N\\
   & C_{ri} - C_{r-1,i} \geqslant T_{ri} & & 2 \leqslant r \leqslant M, 
      1 \leqslant i \leqslant N\\
   & C_{ri} - C_{rk} + PD_{ik} \geqslant T_{ri} & & 1 \leqslant r \leqslant M, 
      1 \leqslant i < k \leqslant N\\
   & C_{ri} - C_{rk} + PD_{ik} \leqslant P - T_{rk} & & 1 \leqslant r \leqslant M, 
      1 \leqslant i < k \leqslant N\\
   & C_\mathrm{max} \geqslant C_{Mi} & & 1 \leqslant i \leqslant N
\end{align}


\section{Resultados computacionais}

\tiny
% latex table generated in R 3.6.1 by xtable 1.8-4 package
% Fri Nov  1 11:27:33 2019
\begin{longtable}{lrrccccr}
\toprule
 Instância & BKS & $\alpha$ & F.O. GRASP & GAP$_\mathrm{GRASP}$ (\%) & F.O G+BL & GAP$_\mathrm{G+BL}$ (\%) & Tempo (s.) \\ 
\midrule
\endhead

\bottomrule
\endfoot


 VFR10\_15\_1 & 1307.00 & 0.00 & $1424 \pm 0$ & 8.95 & $1339.6 \pm 18.319$ & 2.49 & $1.5 \pm 0.04$ \\ 
  VFR10\_15\_1 & 1307.00 & 0.20 & $1431.5 \pm 24.024$ & 9.53 & $1354.2 \pm 23.011$ & 3.61 & $1.4 \pm 0.03$ \\ 
  VFR10\_15\_1 & 1307.00 & 0.40 & $1459.6 \pm 39.884$ & 11.68 & $1364.2 \pm 28.944$ & 4.38 & $1.5 \pm 0.04$ \\ 
  VFR10\_15\_1 & 1307.00 & 0.60 & $1465.8 \pm 43.827$ & 12.15 & $1346.1 \pm 42.331$ & 2.99 & $1.4 \pm 0.03$ \\ 
  VFR10\_15\_1 & 1307.00 & 0.80 & $1470.6 \pm 49.934$ & 12.52 & $1362.9 \pm 30.205$ & 4.28 & $1.5 \pm 0.04$ \\ 
  VFR10\_15\_1 & 1307.00 & 1.00 & $1528.7 \pm 75.588$ & 16.96 & $1342.2 \pm 28.867$ & 2.69 & $1.5 \pm 0.03$ \\ 
   \midrule
VFR100\_60\_1 & 9395.00 & 0.00 & $11247 \pm 0$ & 19.71 & $10008.8 \pm 47.123$ & 6.53 & $57.7 \pm 0.59$ \\ 
  VFR100\_60\_1 & 9395.00 & 0.20 & $11251.8 \pm 118.302$ & 19.76 & $10054.5 \pm 70.099$ & 7.02 & $57.7 \pm 0.42$ \\ 
  VFR100\_60\_1 & 9395.00 & 0.40 & $11243.3 \pm 121.401$ & 19.67 & $10039.1 \pm 54.017$ & 6.86 & $57.9 \pm 0.52$ \\ 
  VFR100\_60\_1 & 9395.00 & 0.60 & $11287.2 \pm 131.908$ & 20.14 & $10040.9 \pm 73.843$ & 6.87 & $58.5 \pm 0.87$ \\ 
  VFR100\_60\_1 & 9395.00 & 0.80 & $11409.9 \pm 164.966$ & 21.45 & $10048.8 \pm 69.904$ & 6.96 & $58 \pm 1$ \\ 
  VFR100\_60\_1 & 9395.00 & 1.00 & $11312.1 \pm 187.334$ & 20.41 & $10057.8 \pm 55.519$ & 7.05 & $58.2 \pm 0.99$ \\ 
   \midrule
VFR20\_10\_3 & 1592.00 & 0.00 & $2017 \pm 0$ & 26.70 & $1687.5 \pm 29.304$ & 6.00 & $2.1 \pm 0.05$ \\ 
  VFR20\_10\_3 & 1592.00 & 0.20 & $2030.4 \pm 44.443$ & 27.54 & $1685.8 \pm 23.223$ & 5.89 & $2 \pm 0.03$ \\ 
  VFR20\_10\_3 & 1592.00 & 0.40 & $1954.6 \pm 51.036$ & 22.78 & $1682 \pm 21.417$ & 5.65 & $2 \pm 0.03$ \\ 
  VFR20\_10\_3 & 1592.00 & 0.60 & $1931 \pm 47.044$ & 21.29 & $1690.8 \pm 39.6$ & 6.21 & $2 \pm 0.04$ \\ 
  VFR20\_10\_3 & 1592.00 & 0.80 & $1894.9 \pm 65.665$ & 19.03 & $1692.3 \pm 32.094$ & 6.30 & $2 \pm 0.02$ \\ 
  VFR20\_10\_3 & 1592.00 & 1.00 & $2007.5 \pm 64.24$ & 26.10 & $1682.7 \pm 24.157$ & 5.70 & $2 \pm 0.04$ \\ 
   \midrule
VFR20\_20\_1 & 2270.00 & 0.00 & $2715 \pm 0$ & 19.60 & $2360.1 \pm 33.478$ & 3.97 & $3.9 \pm 0.07$ \\ 
  VFR20\_20\_1 & 2270.00 & 0.20 & $2759.4 \pm 69.617$ & 21.56 & $2355.8 \pm 41.214$ & 3.78 & $3.9 \pm 0.08$ \\ 
  VFR20\_20\_1 & 2270.00 & 0.40 & $2745.8 \pm 80.5$ & 20.96 & $2350 \pm 25.573$ & 3.52 & $3.9 \pm 0.08$ \\ 
  VFR20\_20\_1 & 2270.00 & 0.60 & $2706.7 \pm 69.72$ & 19.24 & $2376.6 \pm 31.178$ & 4.70 & $3.9 \pm 0.06$ \\ 
  VFR20\_20\_1 & 2270.00 & 0.80 & $2735.3 \pm 44.475$ & 20.50 & $2362.9 \pm 26.236$ & 4.09 & $3.8 \pm 0.05$ \\ 
  VFR20\_20\_1 & 2270.00 & 1.00 & $2787.7 \pm 84.592$ & 22.81 & $2366.9 \pm 38.766$ & 4.27 & $3.9 \pm 0.07$ \\ 
   \midrule
VFR500\_40\_1 & 28548.00 & 0.00 & $33119 \pm 0$ & 16.01 & $30640.6 \pm 67.832$ & 7.33 & $200.4 \pm 8.47$ \\ 
  VFR500\_40\_1 & 28548.00 & 0.20 & $33572.6 \pm 207.304$ & 17.60 & $30753.7 \pm 111.634$ & 7.73 & $200 \pm 4.51$ \\ 
  VFR500\_40\_1 & 28548.00 & 0.40 & $33516.3 \pm 217.696$ & 17.40 & $30697.4 \pm 107.934$ & 7.53 & $197.2 \pm 1.52$ \\ 
  VFR500\_40\_1 & 28548.00 & 0.60 & $33720.7 \pm 278.457$ & 18.12 & $30681.7 \pm 127.513$ & 7.47 & $198.4 \pm 1.59$ \\ 
  VFR500\_40\_1 & 28548.00 & 0.80 & $33710.1 \pm 176.109$ & 18.08 & $30688.4 \pm 101.606$ & 7.50 & $199.6 \pm 3.45$ \\ 
  VFR500\_40\_1 & 28548.00 & 1.00 & $33522.1 \pm 494.424$ & 17.42 & $30741.5 \pm 113.56$ & 7.68 & $200.9 \pm 7.53$ \\ 
   \midrule
VFR500\_60\_3 & 31125.00 & 0.00 & $36930 \pm 0$ & 18.65 & $33539.6 \pm 106.966$ & 7.76 & $298.5 \pm 4.31$ \\ 
  VFR500\_60\_3 & 31125.00 & 0.20 & $36741.8 \pm 257.954$ & 18.05 & $33624.6 \pm 167.947$ & 8.03 & $300.7 \pm 3.79$ \\ 
  VFR500\_60\_3 & 31125.00 & 0.40 & $36508.5 \pm 314.718$ & 17.30 & $33535.1 \pm 81.036$ & 7.74 & $299.2 \pm 3.89$ \\ 
  VFR500\_60\_3 & 31125.00 & 0.60 & $36596.7 \pm 410.012$ & 17.58 & $33576.6 \pm 71.104$ & 7.88 & $300.6 \pm 3.38$ \\ 
  VFR500\_60\_3 & 31125.00 & 0.80 & $36482.3 \pm 288.909$ & 17.21 & $33490.7 \pm 96.158$ & 7.60 & $298.3 \pm 3.3$ \\ 
  VFR500\_60\_3 & 31125.00 & 1.00 & $36327 \pm 354.164$ & 16.71 & $33530.5 \pm 65.58$ & 7.73 & $298.7 \pm 2.61$ \\ 
   \midrule
VFR60\_10\_3 & 3423.00 & 0.00 & $4357 \pm 0$ & 27.29 & $3632.6 \pm 62.45$ & 6.12 & $6 \pm 0.06$ \\ 
  VFR60\_10\_3 & 3423.00 & 0.20 & $4367.2 \pm 83.639$ & 27.58 & $3637.4 \pm 67.612$ & 6.26 & $6 \pm 0.14$ \\ 
  VFR60\_10\_3 & 3423.00 & 0.40 & $4334.5 \pm 93.77$ & 26.63 & $3630.7 \pm 55.041$ & 6.07 & $6 \pm 0.08$ \\ 
  VFR60\_10\_3 & 3423.00 & 0.60 & $4305 \pm 84.374$ & 25.77 & $3608.3 \pm 50.557$ & 5.41 & $5.9 \pm 0.11$ \\ 
  VFR60\_10\_3 & 3423.00 & 0.80 & $4430.9 \pm 112.114$ & 29.44 & $3603.6 \pm 72.537$ & 5.28 & $6 \pm 0.08$ \\ 
  VFR60\_10\_3 & 3423.00 & 1.00 & $4390.9 \pm 136.315$ & 28.28 & $3626.3 \pm 54.214$ & 5.94 & $6 \pm 0.09$ \\ 
   \midrule
VFR60\_5\_10 & 3663.00 & 0.00 & $3849 \pm 0$ & 5.08 & $3668.4 \pm 7.291$ & 0.15 & $3.2 \pm 0.09$ \\ 
  VFR60\_5\_10 & 3663.00 & 0.20 & $3833.2 \pm 22.25$ & 4.65 & $3667.9 \pm 5.971$ & 0.13 & $3.2 \pm 0.13$ \\ 
  VFR60\_5\_10 & 3663.00 & 0.40 & $3847.5 \pm 24.699$ & 5.04 & $3672.2 \pm 8.574$ & 0.25 & $3.1 \pm 0.05$ \\ 
  VFR60\_5\_10 & 3663.00 & 0.60 & $3887.3 \pm 33.002$ & 6.12 & $3674.4 \pm 8.03$ & 0.31 & $3.2 \pm 0.06$ \\ 
  VFR60\_5\_10 & 3663.00 & 0.80 & $3917.5 \pm 60.99$ & 6.95 & $3668.6 \pm 7.152$ & 0.15 & $3.2 \pm 0.03$ \\ 
  VFR60\_5\_10 & 3663.00 & 1.00 & $3914 \pm 87.271$ & 6.85 & $3665.6 \pm 1.897$ & 0.07 & $3.1 \pm 0.05$ \\ 
   \midrule
VFR600\_20\_1 & 31433.00 & 0.00 & $35473 \pm 0$ & 12.85 & $32904.4 \pm 69.306$ & 4.68 & $118.4 \pm 1.86$ \\ 
  VFR600\_20\_1 & 31433.00 & 0.20 & $35828.1 \pm 208.495$ & 13.98 & $32930 \pm 65.09$ & 4.76 & $121.1 \pm 5.56$ \\ 
  VFR600\_20\_1 & 31433.00 & 0.40 & $35865.6 \pm 235.46$ & 14.10 & $32999.7 \pm 123.094$ & 4.98 & $119.3 \pm 1.99$ \\ 
  VFR600\_20\_1 & 31433.00 & 0.60 & $36042.2 \pm 370.703$ & 14.66 & $32982.4 \pm 68.39$ & 4.93 & $119.2 \pm 1.82$ \\ 
  VFR600\_20\_1 & 31433.00 & 0.80 & $36070.7 \pm 419.604$ & 14.75 & $32932.5 \pm 134.142$ & 4.77 & $123.1 \pm 9.14$ \\ 
  VFR600\_20\_1 & 31433.00 & 1.00 & $35964.1 \pm 425.271$ & 14.42 & $32990.1 \pm 97.588$ & 4.95 & $122.6 \pm 7.68$ \\ 
   \midrule
VFR700\_20\_10 & 36417.00 & 0.00 & $40916 \pm 0$ & 12.35 & $37857.4 \pm 114.996$ & 3.96 & $140.6 \pm 2.03$ \\ 
  VFR700\_20\_10 & 36417.00 & 0.20 & $40858.5 \pm 222.473$ & 12.20 & $37792.3 \pm 93.295$ & 3.78 & $140 \pm 3.16$ \\ 
  VFR700\_20\_10 & 36417.00 & 0.40 & $41003.3 \pm 407.564$ & 12.59 & $37865.9 \pm 79.689$ & 3.98 & $139 \pm 2.11$ \\ 
  VFR700\_20\_10 & 36417.00 & 0.60 & $41196.8 \pm 355.395$ & 13.13 & $37798.9 \pm 87.46$ & 3.79 & $142.6 \pm 9.19$ \\ 
  VFR700\_20\_10 & 36417.00 & 0.80 & $41036.6 \pm 281.03$ & 12.69 & $37882.2 \pm 110.235$ & 4.02 & $140.3 \pm 3.43$ \\ 
  VFR700\_20\_10 & 36417.00 & 1.00 & $41214.5 \pm 385.111$ & 13.17 & $37807.6 \pm 124.189$ & 3.82 & $139.8 \pm 2.51$ \\ 




 
\end{longtable}

\normalsize

\begin{table}[ht]
   \begin{tabular}{lrrrr}
      \toprule
      Instância & BKS & Valor relaxação & Obj. solução inteira & GAP$_\mathrm{BKS}$ (\%) \\
      \midrule
      VFR10\_15\_1 & 1307 & 880.0 & 1307 & \phantom{0}0.0\\
      VFR10\_10\_3 & 1592 & 687.0 & 1873 & 56.9\\
      VFR\_20\_20\_1 & 2270 & 1391.0 & 2573 & 42.6\\
      VFR60\_5\_10 & 3663 & 382.0 & 3878 & 89.3\\
      VFR100\_60\_1 & 9395  &  TL & -- & $\infty$\\
      VFR500\_40\_1 & 28548 &  TL & -- & $\infty$\\
      VFR500\_60\_3 & 31125 &  TL & -- & $\infty$\\
      VFR600\_20\_1 & 31433 &  TL & -- & $\infty$\\
      VFR700\_20\_10 & 36417 & TL & -- & $\infty$\\ 
      \bottomrule
   \end{tabular}
\end{table}

\begin{figure}
   \centering
   \subfloat[][]{
      \includegraphics[width=0.48\textwidth]{../resultados/boxplot-VFR10_15_1.pdf}
   }
   \subfloat[][]{
      \includegraphics[width=0.48\textwidth]{../resultados/boxplot-VFR20_10_3.pdf}
   }\\
   \subfloat[][]{
      \includegraphics[width=0.48\textwidth]{../resultados/boxplot-VFR20_20_1.pdf}
   }
   \subfloat[][]{
      \includegraphics[width=0.48\textwidth]{../resultados/boxplot-VFR60_5_10.pdf}
   }\\
   \subfloat[][]{
      \includegraphics[width=0.48\textwidth]{../resultados/boxplot-VFR100_60_1.pdf}
   }
   \subfloat[][]{
      \includegraphics[width=0.48\textwidth]{../resultados/boxplot-VFR100_60_1.pdf}
   }
   \caption{Boxplot relacionando valor médio da função objetivo para as diversas 
      instâncias de testes, com vários valores \bm{$\alpha$} e 10 replicações
      por caso de teste.}
\end{figure}

\begin{figure}
   \ContinuedFloat
   \subfloat[][]{
      \includegraphics[width=0.48\textwidth]{../resultados/boxplot-VFR500_40_1.pdf}
   }
   \subfloat[][]{
      \includegraphics[width=0.48\textwidth]{../resultados/boxplot-VFR500_60_3.pdf}
   }\\
   \subfloat[][]{
      \includegraphics[width=0.48\textwidth]{../resultados/boxplot-VFR600_20_1.pdf}
   }
   \subfloat[][]{
      \includegraphics[width=0.48\textwidth]{../resultados/boxplot-VFR700_20_10.pdf}
   }
   \caption{Boxplot relacionando valor médio da função objetivo para as diversas 
      instâncias de testes, com vários valores \bm{$\alpha$} e 10 replicações
      por caso de teste. Continuação da figura anterior.}
\end{figure}

\bibliographystyle{sbc}
\bibliography{referencias}

\end{document}
